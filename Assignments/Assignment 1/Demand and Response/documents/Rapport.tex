\documentclass{article}
\usepackage[utf8]{inputenc}
\usepackage{soul, amsmath, listings, pdfpages}
\usepackage{tikz}
\usetikzlibrary{arrows.meta,positioning}

\lstset{language=Python}          % Set your language (you can change the language for each code-block optionally)


\title{INF5870 - First Mandatory Assignment}
\author{Zahra G. Yndestad \\ Khalil Abuawad \\ Marius E. G. Andresen \\ Christopher A. Trotter}
\date{February 2017}

\begin{document}

\maketitle

\section{Question 1}
	As the question can conclude into multiple examples leading to the same optimal minimum, depending on the assumptions made, we have decided to give a time frame for when these shiftable appliances can run by trying to cover all possible cases and giving an argument for the edge cases. Since all of the mentioned appliances for the home are shiftable, the mention of a setup time and deadline would be redundant. We will rather assume that the total runtime of each appliance is less than 20 hours. Giving us that each appliance can be completed within the off-peak hours. Also, in the assignment it is only provided the daily cost of these various appliances, but not the cost within a time span such as for example the light bulb. Allowing us to suggest any time span that meets the daily consumption by respecting the constraint above.  As a condition, each of the appliance has to have their runtime outside the peak hours to achieve the optimal minimum. Any example which provides the optimal minimum will have to respect this condition, regardless of the strategy. If these strategies are reasonable is what we will further discuss.

    \textbf{Notes:} There will be two different perspectives of strategies to consider which are centralized or distributed. These are what will be described in the next part.


\end{document}

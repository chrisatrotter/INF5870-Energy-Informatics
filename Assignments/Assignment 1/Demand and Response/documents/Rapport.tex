\documentclass{article}
\usepackage[utf8]{inputenc}
\usepackage{soul, amsmath, listings, pdfpages}
\usepackage{tikz}
\usetikzlibrary{arrows.meta,positioning}

\newtheorem{theorem}{Theorem}
\newtheorem{corollary}{Corollary}[theorem]
\newtheorem{lemma}[theorem]{Lemma}
\newtheorem{definition}[theorem]{Defintion}

\lstset{language=Python}          % Set your language (you can change the language for each code-block optionally)

\usepackage[backend=biber,style=numeric,natbib=true]{biblatex} % Use the bibtex backend with the authoryear citation style (which resembles APA)

\addbibresource{References.bib} % The filename of the bibliography

\title{INF5870 - First Mandatory Assignment}
\author{Zahra G. Yndestad \\ Khalil Abuawad \\ Marius E. G. Andresen \\ Christopher A. Trotter}
\date{February 2017}

\begin{document}

\maketitle

\section{Question 1}
	Since the question can derive multiple examples that lead to the same optimal minimum, depending on the assumptions made, we have decided to give a time frame for when these shiftable appliances can run and we shall attempt to cover all possible cases. Since all of the mentioned appliances for the home are shiftable, then providing examples of different setup times and deadlines would be redundant. We will rather assume that the total runtime of each appliance is less than 20 hours. Giving us that each appliance can be completed within the off-peak hours. Also, in the assignment we are provided the daily cost of these various appliances which differs from non-shiftable appliances, for example: the light bulb. Allowing us to suggest any time span that meets the daily consumption.  As a condition, each of the appliances has to have their runtime outside the peak hours to achieve the optimal minimum. Any example which provides the optimal minimum will have to respect this condition, regardless of the strategy. We will provide a general definition of a strategy, then discuss what is a reasonable strategy.
	
	\begin{definition}
    	A strategy is a high level plan to achieve one or more goals under conditions of uncertainty{\cite{wikistrategy}}. 
	\end{definition}
	
	From our assumptions above, we can assume our strategy is trivial in the sense that all the appliances can be run in the off-peak hours. Secondly, all strategies can be viewed from a spectre where all appliances are running at the same time, centralized, to running sequentially in some order, distributed. From these different perspectives of strategies, we can now discuss what is a reasonable strategy by first defining Demand and Response and after taking into consideration general issues of Demand and Response.
	
	\begin{definition}
	    Demand Response Management (DRM) is defined as changes in electric usage by end‐use customers from their normal consumption patterns in response to changes in the price of electricity over time, or to incentive payments designed to induce lower electricity use at times of high wholesale market prices or when system reliability is jeopardized\cite{defDRM}.
	\end{definition}
	
	
	As an assumption to the question, the customer is to be incentivized based on the electrical pricing scheme known as \emph{Time-Of-Use}. Meaning that the customer is informed ahead of time about the price of electricity. Which addresses an underlying assumption that this will result in users using less electricity when electricity prices are high.
	

    
    
    \printbibliography[heading=bibintoc]
\end{document}
